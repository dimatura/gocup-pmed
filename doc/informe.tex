% vim: tw=79 encoding=utf8
\documentclass[11pt,letterpaper]{article}
\usepackage{graphicx}
\usepackage[utf8]{inputenc}  % char encoding
\usepackage[margin=1cm]{geometry}

% \usepackage[T1]{fontenc}
\pagestyle{empty}   % do page numbering ('empty' turns off)
\title{GOCUP 2008/2009 -- Report}
\author{Daniel Maturana (dimatura@puc.cl)}
\begin{document}
\maketitle
\section{Methodology}
I experimented with various different (meta) heuristics, such as genetic
algorithms, estimation of distribution algorthms and variations on greedy
procedures. Given enough time, these methods would tend to yield
the same set of optima, which I give below. 

With these optima in hand, I worked on improving the computational time
required to consistently achieve these solutions. It was determined that the
best method was simple greedy hill climbing via ``vertex substitution'' and
random restart, i.e. an initial random solution is greedily improved via swaps
of cities between the set of cities considered as medians and non-medians until
a local minima in cost is reached. The procedure is repeated until 28
consecutive runs fail to improve the best solution or 82 times in total,
whichever happens first. These values were chosen to result in the optima given
below with approximately .95 probability, but may be specified by the user. 

The algorithm was implemented in C++ with special attention to performance.
Various of the optimizations suggested by  M. Resende y R. Wernicke ('On the
implementation of a swap based local search procedure for the p-median
problem', Proceedings of the fifth workshop on Algorithm Engineering, 2003)
were used.

On an 2GHz Intel Pentium IV with 2GB of RAM 82 independent runs of the
algorithm take approximately 6 seconds, though the algorithm often ends
earlier. (It can't be predicted, given its stochastic nature). Therefore 
the complete set of optima can usually be computed in less than 20
seconds.

\section{Results}
(BONUS: see included png files for a visualization!)

\begin{enumerate}

\item Medians: [99, 502, 312, 279, 802, 472, 133, 796, 105, 876]
Cost: 6203859

\item Medians: [663, 217, 28, 825, 791, 932, 456, 261, 206, 755]
Cost: 6137625

\item Medians: [908, 965, 652, 46, 852, 615, 49, 863, 54, 411]
Cost: 6273000

\item Medians: [417, 208, 306, 439, 435, 67, 425, 976, 945, 662]
Cost: 6182962

\item Sum
24797446
\end{enumerate}

\section{Compilation and usage}
The program was developed in C++ using GNU g++ in Ubuntu Linux Hardy Heron. It was
also verified to compile and run under Windows XP using the GNU MingW toolchain. 
It has no external library dependencies. To compile, simply run {\tt make}.
The program is named {\tt pmedian} and accepts various arguments. The simplest
usage is {\tt pmedian <number>} where number is the problem to solve. It
includes various more options which can are documented in the help message
which is displayed with {\tt pmedian -h}.

\end{document}
